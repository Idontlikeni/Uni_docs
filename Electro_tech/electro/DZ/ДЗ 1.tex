% Options for packages loaded elsewhere
\PassOptionsToPackage{unicode}{hyperref}
\PassOptionsToPackage{hyphens}{url}
%
\documentclass[
]{article}
\usepackage{amsmath,amssymb}
\usepackage{iftex}
\ifPDFTeX
  \usepackage[T1]{fontenc}
  \usepackage[utf8]{inputenc}
  \usepackage{textcomp} % provide euro and other symbols
\else % if luatex or xetex
  \usepackage{unicode-math} % this also loads fontspec
  \defaultfontfeatures{Scale=MatchLowercase}
  \defaultfontfeatures[\rmfamily]{Ligatures=TeX,Scale=1}
\fi
\usepackage{lmodern}
\ifPDFTeX\else
  % xetex/luatex font selection
\fi
% Use upquote if available, for straight quotes in verbatim environments
\IfFileExists{upquote.sty}{\usepackage{upquote}}{}
\IfFileExists{microtype.sty}{% use microtype if available
  \usepackage[]{microtype}
  \UseMicrotypeSet[protrusion]{basicmath} % disable protrusion for tt fonts
}{}
\makeatletter
\@ifundefined{KOMAClassName}{% if non-KOMA class
  \IfFileExists{parskip.sty}{%
    \usepackage{parskip}
  }{% else
    \setlength{\parindent}{0pt}
    \setlength{\parskip}{6pt plus 2pt minus 1pt}}
}{% if KOMA class
  \KOMAoptions{parskip=half}}
\makeatother
\usepackage{xcolor}
\setlength{\emergencystretch}{3em} % prevent overfull lines
\providecommand{\tightlist}{%
  \setlength{\itemsep}{0pt}\setlength{\parskip}{0pt}}
\setcounter{secnumdepth}{-\maxdimen} % remove section numbering
\usepackage{bookmark}
\IfFileExists{xurl.sty}{\usepackage{xurl}}{} % add URL line breaks if available
\urlstyle{same}
\hypersetup{
  hidelinks,
  pdfcreator={LaTeX via pandoc}}

\author{}
\date{}

\begin{document}

\section{Часть 1}\label{ux447ux430ux441ux442ux44c-1}

\textbf{Дано:}

\$\$

\displaylines{R_{1} = 15,\\R_{2} = 28,\\R_{3} = 17,\\R_{4} = 62,\\R_{5} = 38,\\R_{6} = 21,\\E_{1} = 13,\\E_{4} = 14}

\$\$ !{[}{[}Pasted image 20240916200657.png{]}{]} \#\# 1.1 Определить
токи во всех ветвях методом непосредственного законов Кирхгофа

узел a: \(-I_{1} - I_{4} - I_{6} = 0\) узел b:
\(I_{1} + I_{2} + I_{3} = 0\) узел c: \(-I_{3} + I_{5} + I_{6} = 0\)

контур abdma: \(I_{1}R_{1} - I_{2}R_{2} - I_{4}R_{4} = E_{1} - E_{4}\)
контур dbcd: \(I_{2}R_{2} - I_{3}R_{3} - I_{5}R_{5} = 0\) контур amdcna:
\(I_{4}R_{4} + I_{5}R_{5} - I_{6}R_{6} = E_{4}\) Получается система
уравнений

\[
\begin{cases}
-I_{1} - I_{4} - I_{6} = 0 \\ I_{1} + I_{2} + I_{3} = 0 \\ -I_{3} + I_{5} + I_{6} = 0 \\ I_{1}R_{1} - I_{2}R_{2} - I_{4}R_{4} = E_{1} - E_{4} \\ I_{2}R_{2} - I_{3}R_{3} - I_{5}R_{5} = 0 \\ I_{4}R_{4} + I_{5}R_{5} - I_{6}R_{6} = E_{4}
\end{cases}
\] Решим систему уравнений с помощью матриц, представив систему в виде:
(R)(I) = (E). \[
\begin{pmatrix}
-1 & 0 & 0 & -1 & 0 & -1 \\
1 & 1 & 1 & 0 & 0 & 0 \\
0 & 0 & -1 & 0 & 1 & 1 \\
15 & -28 & 0 & -62 & 0 & 0 \\
0 & 28 & -17 & 0 & -38 & 0 \\
0 & 0 & 0 & 62 & 38 & -21
\end{pmatrix}
\begin{pmatrix}
I_{1} \\
I_{2} \\
I_{3} \\
I_{4} \\
I_{5} \\
I_{6}
\end{pmatrix}
 = 
\begin{pmatrix}
0 \\
0 \\
0 \\
13 - 14 \\
0 \\
14
\end{pmatrix}
\] После решения данной системы, получим следующие значения токов: \[
\begin{pmatrix}
I_{1} \\
I_{2} \\
I_{3} \\
I_{4} \\
I_{5} \\
I_{6}
\end{pmatrix} = 
\begin{pmatrix}
  78341/357043 \\
  -3921/357043 \\
 -74420/357043 \\
  26483/357043 \\
  30404/357043 \\
-104824/357043
\end{pmatrix}
 = 
 \begin{pmatrix}
0.2194 \\
-0.010982 \\
-0.2084 \\
0.074173 \\
0.085155 \\
-0.2936
\end{pmatrix}
\] \$\$

\displaylines{I_{1} = 0.2194;\\I_{2} = -0.010982;\\I_{3} = -0.2084;\\I_{4} = 0.074173;\\I_{5} = 0.085155;\\I_{6} = -0.2936}

\$\$ \#\# 1.2 Составить баланс мощностей

Уравнение баланса мощностей для цепи постоянного тока имеет вид: \[
\sum_{n}R_{n}I_{n}^2 = \sum_{k}\pm I_{k}E_{k}
\] \[
R_{1}*I_{1}^2 + R_{2}*I_{2}^2 + R_{3}*I_{3}^2 + R_{4}*I_{4}^2 + R_{5}*I_{5}^2 + R_{6}*I_{6}^2 = I_{4}E_{4} + I_{6}E_{6}
\] Подставив числа, получаем: \[\begin{gather}
15 * 0.2194 ^ 2 + 28 * (-0.010982)^2 + 17 * (-0.2084)^2 + \\ + 62 * 0.074173^2 + 38 * 0.085155^2 + 21 * (-0.2936)^2 =\\= 13 * 0.2194 + 14 * 0.074173
\\
3,8906 = 3,8906
\end{gather}\] Баланс сошелся

\section{Часть 2}\label{ux447ux430ux441ux442ux44c-2}

По закону контурных токов, надо найти контурные токи, а затем по ним
рассчитать токи в ветвях. Система уравнений для контурных токов:

Представим систему в матричном виде \[
\begin{pmatrix}
R_{1} + R_{2} + R_{4} & -R_{2} & -R_{4} \\
-R_{2} & R_{2} + R_{3} + R_{5} & -R_{5} \\
-R_{4} & -R_{5} & R_{4} + R_{5} + R_{6}
\end{pmatrix}
\begin{pmatrix}
I_{11} \\
I_{22} \\
I_{33}
\end{pmatrix} = 
\begin{pmatrix}
E_{1} - E_{4} \\
0 \\
E_{4}
\end{pmatrix}
\] Решаем, получается: \[
\begin{pmatrix}
I_{11} \\
I_{22} \\
I_{33}
\end{pmatrix} = 
\begin{pmatrix}
0.2194\\
0.2084\\
0.2936\\
\end{pmatrix}
\] Находим значения токов в ветвях: \[
\begin{gather}
I_{1} = I_{11} = 0.2194\\
I_{2} = I_{22} - I_{11} = 0.2084 - 0.2194\\
I_{3} = -I_{22} = -0.2084\\
I_{4} = I_{33} - I_{11} = 0.2936 - 0.2194\\
I_{5} = I_{33} - I_{22} = 0.2936 -0.2084\\
I_{6} = -I_{33} = -0.2936
\end{gather}
\] Получаем: \[
\begin{gather}
I_{1} = I_{11} = 0.2194\\
I_{2} = -0.011000\\
I_{3} = -0.2084\\
I_{4} = 0.074200\\
I_{5} = 0.085200\\
I_{6} = -0.2936
\end{gather}
\] Ответы сходятся

\end{document}
